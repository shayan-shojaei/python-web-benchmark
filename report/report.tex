\documentclass[12pt,onecolumn,a4paper,titlepage]{article}
\usepackage{blindtext}
\usepackage{amsmath}
\usepackage{color,xcolor}
\usepackage{listings}
\usepackage{hyperref}
\usepackage{multirow}


\definecolor{codegreen}{rgb}{0,0.6,0}
\definecolor{codegray}{rgb}{0.5,0.5,0.5}
\definecolor{codepurple}{rgb}{0.58,0,0.82}
\definecolor{backcolour}{rgb}{0.95,0.95,0.92}

\lstdefinestyle{mystyle}{
    backgroundcolor=\color{backcolour},   
    commentstyle=\color{codegreen},
    keywordstyle=\color{magenta},
    numberstyle=\tiny\color{codegray},
    stringstyle=\color{codepurple},
    basicstyle=\ttfamily\footnotesize,
    breakatwhitespace=false,         
    breaklines=true,                 
    captionpos=b,                    
    keepspaces=true,                 
    numbers=left,                    
    numbersep=5pt,                  
    showspaces=false,                
    showstringspaces=false,
    showtabs=false,                  
    tabsize=2
}
\lstset{style=mystyle}

\begin{document}
\title{Python's Web Frameworks: A Comparison} 
\author{Shayan Shojaei}
\date{Spring 2024}
\maketitle

\tableofcontents

\pagebreak

\section{Introduction}
This report presents a comparative analysis of three prominent web
frameworks --Django, Flask, and FastAPI—by implementing a basic CRUD
(Create, Read, Update, Delete) application and evaluating their
performance across two different database systems. The study aims
to provide insights into the efficiency, scalability, and ease of
use of each framework. Django, known for its comprehensive features
and built-in administrative interface, is contrasted with Flask's
lightweight and modular design, and FastAPI's modern, asynchronous
capabilities. Performance metrics such as response time, throughput,
and resource utilization are analyzed under varying workloads to
determine the frameworks' suitability for different application
scenarios. This evaluation offers valuable guidance for developers
in selecting the appropriate framework based on specific project
requirements and performance criteria.


\pagebreak

\section{Overview}
In this study, APIs were implemented and deployed using Django,
Flask, and FastAPI, adhering strictly to the guidelines provided
in each framework's official documentation. No additional optimizations
were applied, ensuring a straightforward and minimalistic approach for

\subsection*{Django}

For Django, a new project was created using the \lstinline{django-admin startproject}
command, followed by the creation of two apps for MongoDB and PostgreSQL respectively,
with \lstinline{python manage.py startapp}. Models were defined in the \lstinline{models.py} file and
URL routing was configured in \lstinline{urls.py}, and the built-in development server
was used for deployment.

\subsection*{Flask}

In Flask, the application was structured by creating a module for Postgresql and MongoDB each, with
routes defined using the \lstinline{@app.get}, etc. decorator. SQLAlchemy was used for
postgres interactions, motor for MongoDB, and the CRUD operations were implemented directly within
the route functions. The application was run using Flask’s built-in server by
executing the script with \lstinline{flask run main.py} which is a WSGI server and not a ASGI server,
resulting in lower performance than a ASGI server.

\subsection*{FastAPI}

FastAPI’s implementation involved defining route functions with Python’s
async capabilities and using Pydantic models for data validation. The
fastapi package was employed to create the app, and CRUD routes were
defined using the \lstinline{@app.get}, \lstinline{@app.post}, \lstinline{@app.put}, and
\lstinline{@app.delete} decorators. The application was deployed using
FastApi's own implementation of an ASGI server, by running \lstinline{fastapi run app.py}.

\pagebreak

\section{Benchmarks}

The benchmarks for evaluating Django, Flask, and FastAPI were set up using
Grafana’s K6, a modern load testing tool. Five distinct types of tests were
designed to comprehensively assess the performance of each framework. The
"create" test involved generating as many items as possible using the POST
method within a 30-second window, focusing on the frameworks' ability to
handle high-volume data insertion. The "read" test measured the efficiency
of each framework in retrieving a single record repeatedly over 30 seconds.
Similarly, the "update" test evaluated the frameworks' performance in modifying
a record multiple times within the same time constraint. The "crud" test was
more complex, involving the creation, reading, updating, and deletion of a
record in succession, aiming to simulate a typical workload scenario within
the 30-second limit. Lastly, the "crud ramp up" test incrementally increased
the number of virtual users from 20 to 100 over 30 seconds, then maintained
this load for an additional 1 minute and 30 seconds, followed by a 20-second
ramp-down period. This test was designed to assess how each framework handles
increasing and sustained loads, providing insights into scalability and
robustness under stress.

\pagebreak

\section{Reports}

\subsection{Create}

\begin{lstlisting}[style=mystyle]
    POST /<database>/
\end{lstlisting}

{\renewcommand{\arraystretch}{1.5}
\begin{tabular}{| p{1.8cm} | p{2cm} | p{1.3cm} | p{2cm} | p{1.9cm} | p{2cm} |} 
    \hline
    Database & Framework & \# VUs & \# \footnotesize{Iterations} & RPS & \footnotesize{Failure \%} \\
    \hline
    \multirow{6}{*}{MongoDB} & \multirow{2}{*}{Django} & 10 & 266 & 8.692 & 0\\
    & & 100 & 1606 & 49.741 & 0\\
    & \multirow{2}{*}{Flask} & 10 & 11169 & 372.000 & 0\\
    & & 100 & 10823 & 357.103 & 0\\
    & \multirow{2}{*}{FastAPI} & 10 & 39130 & 1304.079 & 0\\
    & & 100 & 83077 & 2768.474 & 0\\
    \hline
    \multirow{6}{*}{Postgresql} & \multirow{2}{*}{Django} & 10 & 12711 & 423.400 & 0\\
    & & 100 & 10468 & 345.937 & 0\\
    & \multirow{2}{*}{Flask} & 10 & 16217 & 495.700 & 0\\
    & & 100 & 16440 & 448.550 & 0.004\\
    & \multirow{2}{*}{FastAPI} & 10 & 3309 & 110.152 & 0\\
    & & 100 & 13306 & 442.494 & 0\\
    \hline
\end{tabular}
}

\pagebreak
\subsection{Read}

\begin{lstlisting}[style=mystyle]
    GET /<database>/<id>
\end{lstlisting}

{\renewcommand{\arraystretch}{1.5}
\begin{tabular}{| p{1.8cm} | p{2cm} | p{1.3cm} | p{2cm} | p{1.9cm} | p{2cm} |} 
    \hline
    Database & Framework & \# VUs & \# \footnotesize{Iterations} & RPS & \footnotesize{Failure \%} \\
    \hline
    \multirow{6}{*}{MongoDB} & \multirow{2}{*}{Django} & 10 & 590 & 19.415 & 0\\
    & & 100 & 3842 & 122.790 & 0\\
    & \multirow{2}{*}{Flask} & 10 & 16272 & 484.076 & 0.0005\\
    & & 100 & 16352 & 411.493 & 0.005\\
    & \multirow{2}{*}{FastAPI} & 10 & 113067 & 3768.711 & 0\\
    & & 100 & 121802 & 4057.567 & 0\\
    \hline
    \multirow{6}{*}{Postgresql} & \multirow{2}{*}{Django} & 10 & 12783 & 425.469 & 0\\
    & & 100 & 10896 & 361.387 & 0\\
    & \multirow{2}{*}{Flask} & 10 & 12442 & 319.839 & 0\\
    & & 100 & 16443 & 418.668 & 0.004\\
    & \multirow{2}{*}{FastAPI} & 10 & 13147 & 437.808 & 0\\
    & & 100 & 25608 & 852.717 & 0\\
    \hline
\end{tabular}
}

\pagebreak
\subsection{Update}

\begin{lstlisting}[style=mystyle]
    PUT /<database>/<id>
\end{lstlisting}

{\renewcommand{\arraystretch}{1.5}
\begin{tabular}{| p{1.8cm} | p{2cm} | p{1.3cm} | p{2cm} | p{1.9cm} | p{2cm} |} 
    \hline
    Database & Framework & \# VUs & \# \footnotesize{Iterations} & RPS & \footnotesize{Failure \%} \\
    \hline
    \multirow{6}{*}{MongoDB} & \multirow{2}{*}{Django} & 10 & 336 & 10.843 & 0\\
    & & 100 & 2353 & 73.926 & 0\\
    & \multirow{2}{*}{Flask} & 10 & 10290 & 342.723 & 0\\
    & & 100 & 10317 & 340.439 & 0\\
    & \multirow{2}{*}{FastAPI} & 10 & 33662 & 1121.816 & 0\\
    & & 100 & 105412 & 3503.162 & 0\\
    \hline
    \multirow{6}{*}{Postgresql} & \multirow{2}{*}{Django} & 10 & 10753 & 358.242 & 0\\
    & & 100 & 9676 & 321.159 & 0\\
    & \multirow{2}{*}{Flask} & 10 & 16247 & 450.784 & 0\\
    & & 100 & 16362 & 408.542 & 0.00006\\
    & \multirow{2}{*}{FastAPI} & 10 & 3509 & 116.634 & 0\\
    & & 100 & 8052 & 267.447 & 0\\
    \hline
\end{tabular}
}

\pagebreak
\subsection{CRUD}

\begin{lstlisting}[style=mystyle]
    POST /<database>/
    GET  /<database>/<id>
    PUT  /<database>/<id>
    DEL  /<database>/<id>
\end{lstlisting}

{\renewcommand{\arraystretch}{1.5}
\begin{tabular}{| p{1.8cm} | p{2cm} | p{1.3cm} | p{2cm} | p{1.9cm} | p{2cm} |} 
    \hline
    Database & Framework & \# VUs & \# \footnotesize{Iterations} & RPS & \footnotesize{Failure \%} \\
    \hline
    \multirow{6}{*}{MongoDB} & \multirow{2}{*}{Django} & 10 & 86 & 10.248 & 0\\
    & & 100 & 600 & 68.080 & 0\\
    & \multirow{2}{*}{Flask} & 10 & 3247 & 478.994 & 0\\
    & & 100 & 3303 & 408.065 & 0\\
    & \multirow{2}{*}{FastAPI} & 10 & 12976 & 2161.176 & 0\\
    & & 100 & 24026 & 3992.831 & 0\\
    \hline
    \multirow{6}{*}{Postgresql} & \multirow{2}{*}{Django} & 10 & 3116 & 414.551 & 0\\
    & & 100 & 2499 & 327.052 & 0\\
    & \multirow{2}{*}{Flask} & 10 & 3252 & 444.073 & 0.0004\\
    & & 100 & 3301 & 406.211 & 0.0038\\
    & \multirow{2}{*}{FastAPI} & 10 & 1240 & 205.847 & 0\\
    & & 100 & 3573 & 589.888 & 0\\
    \hline
\end{tabular}
}

\pagebreak
\subsection{CRUD -- Ramp Up}

Same as CRUD, but with a ramp-up of 20 to 100 VUs over 30 seconds, then maintaining 100 VUs for 1 minute and 30 seconds, followed by a 20-second ramp-down.

\vspace*{1cm}

{\renewcommand{\arraystretch}{1.5}
\begin{tabular}{| p{1.8cm} | p{2cm} | p{2cm} | p{1.9cm} | p{2cm} |} 
    \hline
    Database & Framework & \# \footnotesize{Iterations} & RPS & \footnotesize{Failure \%} \\
    \hline
    \multirow{3}{*}{MongoDB} & Django & 1366 & 38.018 & 0\\
    & Flask & 9813 & 330.328 & 0\\
    & FastAPI & 98488 & 3517.344 & 0\\
    \hline
    \multirow{3}{*}{Postgresql} & Django & 10571 & 301.974 & 0\\
    & Flask & 13070 & 405.767 & 0.0019\\
    & FastAPI & 20884 & 745.829 & 0\\
    \hline
\end{tabular}
}

\pagebreak

\section{Analysis}

The analysis of the performance benchmarking reports reveals several important findings. The combination of MongoDB with Django emerged as the worst performer, while the pairing of MongoDB with FastAPI proved to be the best. Despite these results, PostgreSQL should not be discounted, as its capabilities extend beyond mere read and write performance, offering valuable features that may be crucial depending on the application's requirements.

Flask encountered unexpected bottlenecks when interfacing with PostgreSQL, suggesting potential areas for performance improvement. However, the benchmarking aimed to evaluate the frameworks in their default configurations without optimizations. Consequently, this investigation did not delve into resolving these issues. Notably, Flask was the only framework to experience failed requests. Although the number of failures was minimal, the absence of such failures in other frameworks casts doubt on Flask's reliability in comparison.

A gradual increase in the number of virtual users (VUs) slightly enhanced the performance of PostgreSQL when combined with FastAPI and Django, which is consistent with the behavior of PostgreSQL's connection pooling. Conversely, Django did not perform well with MongoDB. This may be attributed to the use of a third-party library for MongoDB integration, as well as Django's design philosophy of abstracting database interactions, which limits developer control over connection management.

The performance disparity between Django and Flask, both using WSGI servers, and FastAPI, which uses an ASGI server out-of-the-box, is notable. While it is possible to deploy Django and Flask with an ASGI server, doing so would require additional setup and middleware, which was not within the scope of these benchmarks. This intrinsic difference in server architecture likely contributes to FastAPI's superior performance in the tests conducted.

\pagebreak

\section{Conclusion}

In conclusion, the benchmarking results indicate that FastAPI with MongoDB provides the best performance among the tested combinations, while Django with MongoDB performs the worst. Although PostgreSQL's performance was not the highest, its robust feature set makes it a viable option for applications that require advanced database functionalities. The unexpected bottlenecks encountered by Flask with PostgreSQL suggest room for optimization, but Flask's reliability issues, as evidenced by failed requests, undermine its competitiveness. The inherent performance advantage of FastAPI, due to its ASGI server architecture, highlights the impact of server choice on framework efficiency. These findings underscore the importance of considering both performance metrics and specific application needs when selecting a web framework and database combination.

\end{document}
